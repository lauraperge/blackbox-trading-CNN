\documentclass[11pt, a4paper]{article}
\usepackage[english, science, titlepage]{ku-frontpage}
\usepackage[utf8]{inputenc}

\usepackage{cite}
\usepackage{natbib, apalike, url}

\setlength\arraycolsep{2 pt}
\setcounter{tocdepth}{2}
\setcounter{secnumdepth}{0}

\assignment{Master Thesis}
\author{Laura Perge}

\title{Time Series Classification with CNN:}
\subtitle{ Automated Trading by Pattern Recognition}
\date{Handed in: \today}
\advisor{Advisors: Rolf Poulsen, Kenneth H. M. Nielsen, Lasse Bøhling}
%\frontpageimage{example.png}

%% spellcheck-language "en"

\begin{document}
\maketitle

\tableofcontents

\begin{abstract}
    Something.
\end{abstract}

\section{Introduction}
Trading on financial markets has been around for more than 200 years and with time, the internet, and technological advancement it has changed and evolved into what it is today:  
a versatile, international, online, easy-to-access, automated and truly immense beast. Its evolution is not over however, and the mentioned features make it the perfect subject 
of artificial intelligence and machine learning. 
Employing algorithms in trading for calculation of asset prices goes back to the beginning of the 20th century and in the early 1950s Harry Markowitz already brought computational finance 
to existence in the pursuit of portfolio optimization. That time the computational resources were inadequate to efficiently utilize these algorithms in trading. From the 1970s to 1990s, 
large-scale computerization, the introduction of PCs, the internet and then amongst other systems the ECN (Electronic Communication Network) changed the game. 

Using automated computer algorithms to execute trading orders has become popular in the  


\section{Related Work}

references to main literatures
\subsection{Financial Time Series Analysis}
\subsection{Algorithmic Trading}
More traditional ones to new AI based ones.
\subsection{Time Series Pattern Recognition}
\subsection{AI \& Deep Learning}


\section{Data and Methodology}
\subsection{Data Overview}
Overview of datasets of different frequencies, asset classes that I use
Data cleaning steps
What data I need
Data transformation steps 

\subsection{Time Series: 1D to 2D}
Idea of transformation
GAF, RP (arvix.org/pdf/1710.00886.pdf 3.1 subsec)
Show image outputs of different types
write about them 

\subsection{Image Labelling}
Labelling: don't forget discrepancy in labelling algo vs image labels
Labelling algo: depends on window size

\subsection{Deep CNN: Image Classification}
How to classify the images, why CNN, \dots, 
Introduce your base model
Model Build, \dots

\section{Evaluation and Results}
\subsection{Classification Performance}
confusion matrix, accuracy measurements, etc.

\subsection{Financial Performance}
description of trading according to model, measurements and ratios
numbers (maybe against other trading algo)

\subsection{Time Consumption}
How long it takes to train, how long it takes to predict

\section{Conclusion}

\section{Appendices}
Calculations, explanations
Hyperparameter optimization Results
Other optimization results

\bibliography{reference}
\bibliographystyle{apalike}

\end{document}
